\documentclass[]{k-cv}

\defaultfontfeatures{Mapping=tex-text}
\usepackage{xcolor}
\addbibresource{bibliography.bib}

\begin{document}
\header{manuel}{montoya}
       {Full stack RoR/JS developer}

% In the aside, each new line forces a line break
\begin{aside}
  \section{contact}
    \href{mailto:mmontoya@gmail.com}{mmontoya@gmail.com}
    Tel  52 (55) 4168-7570
    Cell 52 (55) 2878-3910
    \href{http://github.com/aarkerio}{github://aarkerio}
    \href{https://medium.com/@mmontoya}{medium://mmontoya}
    skype://aarkerio
    Mexico City
  \section{languages}
    native spanish
	  spoken english
	  can read german

  \section{gems I use often}
    capistrano \textbf{factory}-boot \textbf{r}spec
    \textbf{bootstrap} sidekiq unicorn \textbf{hamlit} sassrails \textit{compass}
    decent\_exposure \textbf{debugger} carrierwave \textbf{devise} cells
    cucumber \textbf{guard} capybara \textbf{unicorn}
    \textbf{rubo}cop \textbf{react-rails}
    \textbf{will\_paginate} \textbf{database}\_cleaner

\end{aside}

\section{profile}

I'm a remote web developer with a lot of experience in e-Commerce, e-Learning, e-Health and Fintech sectors.
Last years I've been working remotely for US and Swedish based companies. Coding clean and good software
can be easy if you know your tools and the domain, not just the language. In software, simplicity matters.

I love coffee, solving problems, learning new stuff and hiking with my dogs. I've been
programming for eighteen fun years, since the old days when the Web was made with CGI
pages and ugly java applets. Now I develop using RoR, Clojure and TypeScript with SQL and
noSQL DBs. I code under the TDD and Continuous Integration paradigms. I'm a big fan of
Functional Programming and immutability. My daily desktop is i3 on Debian. I use NetBSD
on my hobby servers because compiling from the sources is fun. \href{https://github.com/aarkerio/ZentaurLMS/blob/development/src/cljs/zentaur/events.cljs}{http://Zentaur}
is my current side project, a LMS built with GraphQL and ClojureScript.

As any real man, I use Emacs as my default editor ;-)

\section{summary of qualifications}
  \begin{itemize}
    \item Eighteen years of experience delivering non-trivial Web solutions, from Single Page Applications
          with React or a new SCORM player, to implementation of Archetype-based medical semantics on e-Health.
    \item Advanced OOP level with Ruby, JavaScript and PHP.
    \item Three years of experience with Amazon Web Services: EC2, DynamoDB, RDS, SQS, S3, API Gateway, Cognito and Lambdas.
    \item Lots of experience installing and configuring server side services (HTTP server,
          RDBMS, RabbitMQ, Subversion, Git, SMTP, SSH, FTP, DNS, LDAP ) on Linux and
          BSD servers.
    \item Aware that good communication, respect and close collaboration inside a team
          are very important to create great products.
    \item Good and patient helping and encouraging the junior developers.
   \end{itemize}

\section{technologies I use often}

    Linux (since 1997), RoR (since 2008), JavaScript (since 1999), Lisp, Perl, jQuery, Node.js, ReactJS, Redux, Webpack, Git,
    Github, Clojure, Apache, Nginx, PostgreSQL, MySQL, SQLite, Redis, Solaris,
    tmux, JSON, SSH, XML, RESTful architecture, GraphQL, Test Driven Development (rspec),
    Integration Server (Jenkins), I develop under Scrum and Kanban methodologies.

\section{education}

\begin{entrylist}
  \entry
    {1995 - 2000}
    {B. Sc. in Computer Science}
    {National University (UNAM)}
	  {Faculty of Sciences, Mexico City}
  \entry
    {June 2008}
    {OOP Design Patterns with Python}
    {JBI Training}
    {London, England}
   \entry
    {December 2010}
    {Refactoring your Rails code}
    {Global Day of Code Retreat}
    {Leipzig, Germany}

\end{entrylist}

\newpage

\section{professional experience}

  \textbf{Clojure/ClojureScript/AWS Developer}  \textit{Not a Bank}
  {\color{gray} {\small Jul. 2018 - Jan. 2020 \par}}
  \begin{itemize}
    \item \textit{Not a Bank} is a startup that offers financial services trough a Cell phone App.
    \item After the definition of the domain, I made the analysis and design of the GraphQL API that the new App
          requires in order to make transactions and notifications.
    \item Together, two Clojure developers and I deployed on production the code for the \linebreak
          payments/report services.
    \item Nine services were deployed as Java AWS Lambdas.
    \item A little Web report tool, for inner use, was made with ClojureScript.
    \item Cognito, SES, RDS PostgreSQL, SQS, S3 and DymamodDB were used on AWS.
  \end{itemize}

  \textbf{Ruby/ReactJS Developer}  \textit{HOMIE}
  {\color{gray} {\small Sep. 2016 - Jul. 2018 \par}}
  \begin{itemize}
    \item Homie is a startup that provides Real State related services. In this project I
          worked with people from the financial and data science fields to create a new
          \textit{Risk Score Assessment Tool}.
    \item I lead a team of four remote engineers to build and launch in five months the tool
          using Rails, MongoDB, RabbitMQ, React and R language.
    \item Several REST and GraphQL APIs were used in order to receive payments and extract info from social networks.
    \item I developed and deployed several services on AWS.
  \end{itemize}

   \textbf{RoR/JS Developer} \textit{Las Lomas Veterinary Clinic}
   {\color{gray} {\small Sep. 2015 - Sep. 2016 \par}}
   \begin{itemize}
     \item I developed a full Customer Management System from scratch using Webpack/React/Redux in the front end.
     \item Coded a RoR4 JSON API. I used PostgreSQL JSONB options.
     \item Full TDD covering using Rspec and Mocha/Enzyme in the front end.
     \item The Website included payroll, suppliers, appointments and invoice sections.
     \item Developed a new gem to connect the mailing list with MailChimp.
   \end{itemize}

  \textbf{Ruby Developer}  \textit{ePublishing, inc.}
  {\color{gray} {\small Mar. 2014 - Jul. 2015 \par}}
   \begin{itemize}
     \item I created a new RESTful JSON API for a pharmaceutical client with Rails 4 to
           search for medicines and medical supplies. Redis was used to store
           special offers and taxes by state.
     \item Wrote the full integration between Rails 4 and Gigya.com API services (shared
           sessions, comments, social networks).
     \item Developed a Newsletter Manager gem. I imported and normalised 300,000
           scientific articles and abstracts from different sources and formats.
     \item Coded ruby scripts to update and interchange data between several kind of
           databases using JSON files.
  \end{itemize}

   \textbf{CTO} \textit{PCSB, SA de CV}
   {\color{gray} {\small Jun. 2007 - Jan. 2014 \par}}
   \begin{itemize}
     \item Six years leading a team of three developers, a pedagogic planner and a
           graphic designer to create four big medical web sites, including a new tool to
           search, import and display medical resources (video, audio, books, magazines).
     \item I designed and coded a new e-Learning platform (virtual classrooms,
           chatrooms, e-tests) for a Medical Institution using Rails 3 and Postgres. I
           developed and implemented SCORM courses.
     \item Google and Facebook APIs were used to create Docs, Sessions and Calendars
           inside the Rails platform.
   \end{itemize}

   \textbf{Developer and webmaster} \textit{University of Mexico}
   {\color{gray} {\small Sep. 2004 - May 2007 \par}}
   \begin{itemize}
     \item Maintaining and porting legacy code from PHP3 spaghetti code to OOP PHP5 using CakePHP framework.
     \item A budget/purchases system, delivering XML and PDF reports.
   \end{itemize}

   \textbf{Web Developer} \textit{Bank of Commerce (Bancomext)}
   {\color{gray} {\small Jan. 1998 - Aug. 2004 \par}}
   \begin{itemize}
     \item I designed, developed and deployed an Intranet solution using Perl and PHP
           over Solaris servers and Oracle as RDBMS.
     \item Using Perl, Apache and MySQL, I developed a Loan Management System.
     \item Coded a new Web File Sharing System.
   \end{itemize}

\end{entrylist}

\end{document}
